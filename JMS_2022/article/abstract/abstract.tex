\thispagestyle{fancy}

\hypertarget{ruxe9sumuxe9}{%
\section*{Résumé}\label{ruxe9sumuxe9}}
\addcontentsline{toc}{section}{Résumé}

L'analyse des cycles économiques, et en particulier la détection précoce
des points de retournement, est un sujet majeur dans l'analyse de la
conjoncture. Les moyennes mobiles, ou les filtres linéaires, sont
omniprésents dans les méthodes d'extraction de la tendance-cycle et
d'ajustement saisonnier. Au centre de la série, des moyennes mobiles
symétriques sont appliquées\footnote{ C'est-à-dire que pour estimer la
  tendance-cycle à la date \(t\), on utilise autant de points dans le
  passé que dans le futur et le même poids est associé aux observations
  passées et futures.}. Cependant, en raison du manque d'observations
futures, les estimations en temps réel doivent s'appuyer sur des
moyennes mobiles asymétriques. C'est ce qui est par exemple fait dans
les méthodes de désaisonnalisation les plus utilisées, TRAMO-SEATS et
X-13ARIMA, qui prolongent la série sur 1 an par un modèle ARIMA. Les
prévisions étant des combinaisons linéaires du passé, cela revient en
réalité à utiliser des moyennes mobiles asymétriques dont les
coefficients sont optimisés par rapport à la prévision avec une longueur
d'avance.

La construction de moyennes mobiles asymétriques performantes en termes
de fidélité (préservation du signal), de révision, de lissage et de
déphasage (délais dans la détection de points de retournement) est un
sujet de recherche toujours ouvert. Cette étude décrit et compare des
approches récentes pour la construction de moyennes mobiles
asymétriques, utilisées pour l'estimation en temps réel de la
tendance-cycle : filtres polynomiaux locaux
\autocite{proietti2008,GrayThomson1996} ; méthodes basées sur une
optimisation sous contrainte d'une somme pondérée de critères de qualité
des moyennes mobiles \autocite{ch15HBSA,trilemmaWMR2019} ; et filtres
basés sur les espaces de Hilbert à noyau reproduisant (RKHS)
\textcite{dagumbianconcini2008}. Elle montre également comment les
filtres polynomiaux locaux peuvent être étendus pour inclure un critère
de temporalité afin de minimiser le déphasage. Enfin, cette étude montre
qu'il est possible d'établir une approche unificatrice générale qui
permet de reproduire l'ensemble des méthodes étudiées.

La comparaison des méthodes sur séries simulées montre qu'il est
important d'adapter la longueur du filtre à la variabilité de la série.
Même si certains filtres RKHS pourraient donner des résultats
satisfaisant en termes de délais dans la détection des points de
retournement, leur utilisation doit être limitées aux premières
estimations de la tendances-cycles. Lorsque l'on se rapproche de
l'estimation finale (i.e., lorsque beaucoup de points dans le futur sont
connus), d'autres méthodes devraient être utilisées afin de limiter les
révisions. Les filtres RKHS sont par ailleurs sujets à problèmes
d'optimisation. Enfin, lorsque la longueur du filtre est adapté à la
variabilité de la série, chercher à conserver des tendances polynomiales
de degré supérieur à un semble introduire de la variance dans les
estimations (et donc plus de révisions) sans gain significatif en termes
de détection de point de retournement.

Cette étude est reproductible. En particulier, toutes les méthodes
décrites sont implémentées dans le \emph{package} \faIcon{r-project}
\texttt{rjdfilters} (\url{https://github.com/palatej/rjdfilters}) et
tous les codes utilisées sont disponibles sous
\url{https://github.com/AQLT/articles}.

\hypertarget{abstract}{%
\section*{Abstract}\label{abstract}}
\addcontentsline{toc}{section}{Abstract}

This paper describes and compares different approaches to build
asymmetric filters: local polynomials filters, methods based on an
optimization of filters' properties (Fidelity-Smoothness-Timeliness,
FST, approach and a data-dependent filter) and filters based on
Reproducing Kernel Hilbert Space. It also describes how local
polynomials filters can be extended to include a timeliness criterion to
minimize phase shift. All these methods can be seen as a special case of
a general unifying framework to derive linear filters.

This paper shows that, when the length of the filter is adapted to the
variability of the series, constraining asymmetric filters to preserve
constant trends (and not necessarily polynomial ones) reduce revision
error and time lag. Therefore, future studies on the subject can focus
on these filters. Moreover, with RKHS filters some optimisation issues
can occurs and they might lead to erratic estimation. They might be able
to produce satisfying results in terms of phase-shift and revisions, but
they should be avoid for the last estimates of the trend-cycle
component: other methods should then be prefered to reduce revisions
with the final estimates.

All the methods are implemented in the \faIcon{r-project} package
\texttt{rjdfilters} and the results can be easily reproduced. The
programs used, and a web version of this report, are available at
\url{https://github.com/AQLT/articles}.
